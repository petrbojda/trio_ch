%% Generated by Sphinx.
\def\sphinxdocclass{report}
\documentclass[letterpaper,10pt,english]{Reference manual}
\ifdefined\pdfpxdimen
   \let\sphinxpxdimen\pdfpxdimen\else\newdimen\sphinxpxdimen
\fi \sphinxpxdimen=.75bp\relax

\usepackage[utf8]{inputenc}
\ifdefined\DeclareUnicodeCharacter
 \ifdefined\DeclareUnicodeCharacterAsOptional\else
  \DeclareUnicodeCharacter{00A0}{\nobreakspace}
\fi\fi
\usepackage{cmap}
\usepackage[T1]{fontenc}
\usepackage{amsmath,amssymb,amstext}
\usepackage{babel}
\usepackage{times}
\usepackage[Bjarne]{fncychap}
\usepackage[dontkeepoldnames]{sphinx}

\usepackage{geometry}

% Include hyperref last.
\usepackage{hyperref}
% Fix anchor placement for figures with captions.
\usepackage{hypcap}% it must be loaded after hyperref.
% Set up styles of URL: it should be placed after hyperref.
\urlstyle{same}
\addto\captionsenglish{\renewcommand{\contentsname}{Contents:}}

\addto\captionsenglish{\renewcommand{\figurename}{Fig.}}
\addto\captionsenglish{\renewcommand{\tablename}{Table}}
\addto\captionsenglish{\renewcommand{\literalblockname}{Listing}}

\addto\extrasenglish{\def\pageautorefname{page}}

\setcounter{tocdepth}{3}
\setcounter{secnumdepth}{3}


\title{Analytic RCW Tracker}
\date{May 10, 2018}
\release{1.2.1}
\author{CTU. Dept. of Measurement}
\newcommand{\sphinxlogo}{\vbox{}}
\renewcommand{\releasename}{Release}
\makeindex

\begin{document}

\maketitle
\sphinxtableofcontents
\phantomsection\label{\detokenize{index::doc}}



\chapter{Introduction}
\label{\detokenize{intro::doc}}\label{\detokenize{intro:welcome-to-automotive-rear-looking-radar-tracker-s-documentation}}\label{\detokenize{intro:introduction}}
The functions described here are prepared to generate digitalized version of radio signals in both baseband and passband. Resuting signals are stored in form of NumPy arrays.


\section{Dependencies:}
\label{\detokenize{intro:dependencies}}
Python 3.6.2, Numpy 1.12.


\section{Installation}
\label{\detokenize{intro:installation}}
No special operation required. All what needs to be done is to save entire ‘’srcpy’’ folder with its contents and use any function. Once they are called from the ‘’srcpy’’ folder they should work immediatelly, otherwise make sure you have the ‘’srcpy’’ folder included in your python system path.


\chapter{Data Containers Module}
\label{\detokenize{datacontainers:trackpoint-label}}\label{\detokenize{datacontainers::doc}}\label{\detokenize{datacontainers:data-containers-module}}
The module defines all the necessary containers to store, sort, filter, import and export data
either raw radar detections or already processed tracks. Also available here are classes to keep
and manipulate data from a referential DGPS system.

All the containers are based on lists of appropriate points. Lists are inherited from a python’s built-in
class ‘list’ with additional methods. Points are inherited from a basic ‘object’ class.
\phantomsection\label{\detokenize{datacontainers:module-data_containers}}\index{data\_containers (module)}

\section{DetectionPoint}
\label{\detokenize{datacontainers:detectionpoint}}
Defines a class which represents a raw radar detection.
\index{DetectionPoint (class in data\_containers)}

\begin{fulllineitems}
\phantomsection\label{\detokenize{datacontainers:data_containers.DetectionPoint}}\pysiglinewithargsret{\sphinxstrong{class }\sphinxcode{data\_containers.}\sphinxbfcode{DetectionPoint}}{\emph{mcc=0}, \emph{beam=0}, \emph{nodet\_permcc=0}, \emph{trackID=0}, \emph{rng=0.0}, \emph{vel=0.0}, \emph{azimuth=0.0}, \emph{left=True}, \emph{car\_width=1.88}}{}
Creates a point as it was detected by the RADAR.
\begin{quote}\begin{description}
\item[{Return type}] \leavevmode
object

\item[{Parameters}] \leavevmode\begin{itemize}
\item {} 
\sphinxstyleliteralstrong{mcc} (\sphinxstyleliteralemphasis{int}) \textendash{} The state of a Master-Clock-Cycle counter of the RADAR.

\item {} 
\sphinxstyleliteralstrong{beam} (\sphinxstyleliteralemphasis{int}) \textendash{} Identifies to which beam the detection has been assigned by the RADAR.

\item {} 
\sphinxstyleliteralstrong{nodet\_permcc} (\sphinxstyleliteralemphasis{int}) \textendash{} Number of detections for this particular MCC.

\item {} 
\sphinxstyleliteralstrong{trackID} (\sphinxstyleliteralemphasis{int}) \textendash{} ID of the track to which the detection has been assigned if so

\item {} 
\sphinxstyleliteralstrong{rng} (\sphinxstyleliteralemphasis{float}) \textendash{} Range of the detection measured by RADAR

\item {} 
\sphinxstyleliteralstrong{vel} (\sphinxstyleliteralemphasis{float}) \textendash{} Velocity of the detection measured by RADAR

\item {} 
\sphinxstyleliteralstrong{azimuth} (\sphinxstyleliteralemphasis{float}) \textendash{} Azimuth of the detection measured by RADAR

\item {} 
\sphinxstyleliteralstrong{left} (\sphinxstyleliteralemphasis{bool}) \textendash{} TRUE if the detection was measured by the left RADAR, FALSE if by right one

\item {} 
\sphinxstyleliteralstrong{car\_width} (\sphinxstyleliteralemphasis{float}) \textendash{} The width of an EGO car. (is being used to align left and right RADAR measurement)

\end{itemize}

\end{description}\end{quote}
\index{set\_XY() (data\_containers.DetectionPoint method)}

\begin{fulllineitems}
\phantomsection\label{\detokenize{datacontainers:data_containers.DetectionPoint.set_XY}}\pysiglinewithargsret{\sphinxbfcode{set\_XY}}{\emph{x}, \emph{y}}{}
For an existing detection sets its \_x and \_y attributes independently
\begin{quote}\begin{description}
\item[{Parameters}] \leavevmode\begin{itemize}
\item {} 
\sphinxstyleliteralstrong{x} (\sphinxstyleliteralemphasis{float}) \textendash{} value to assign to \_x

\item {} 
\sphinxstyleliteralstrong{y} (\sphinxstyleliteralemphasis{float}) \textendash{} value to assign to \_y

\end{itemize}

\end{description}\end{quote}

\end{fulllineitems}


\end{fulllineitems}



\section{ReferencePoint}
\label{\detokenize{datacontainers:referencepoint}}\index{ReferencePoint (class in data\_containers)}

\begin{fulllineitems}
\phantomsection\label{\detokenize{datacontainers:data_containers.ReferencePoint}}\pysiglinewithargsret{\sphinxstrong{class }\sphinxcode{data\_containers.}\sphinxbfcode{ReferencePoint}}{\emph{mccL=0}, \emph{mccR=0}, \emph{TAR\_dist=0.0}, \emph{TAR\_distX=0.0}, \emph{TAR\_distY=0.0}, \emph{TAR\_velX=0.0}, \emph{TAR\_velY=0.0}, \emph{TAR\_hdg=0.0}, \emph{EGO\_velX=0.0}, \emph{EGO\_velY=0.0}, \emph{EGO\_accX=0.0}, \emph{EGO\_accY=0.0}, \emph{EGO\_hdg=0.0}}{}
Creates a point as delivered by a referential DGPS system.
\begin{quote}\begin{description}
\item[{Return type}] \leavevmode
object

\item[{Parameters}] \leavevmode\begin{itemize}
\item {} 
\sphinxstyleliteralstrong{mccL} (\sphinxstyleliteralemphasis{int}) \textendash{} The state of a Master-Clock-Cycle counter of the left RADAR

\item {} 
\sphinxstyleliteralstrong{mccR} (\sphinxstyleliteralemphasis{int}) \textendash{} The state of a Master-Clock-Cycle counter of the right RADAR

\item {} 
\sphinxstyleliteralstrong{TAR\_dist} (\sphinxstyleliteralemphasis{float}) \textendash{} Target vehicle’s distance from the EGO car

\item {} 
\sphinxstyleliteralstrong{TAR\_distX} (\sphinxstyleliteralemphasis{float}) \textendash{} Target vehicle’s distance from the EGO car projected to the X axis

\item {} 
\sphinxstyleliteralstrong{TAR\_distY} (\sphinxstyleliteralemphasis{float}) \textendash{} Target vehicle’s distance from the EGO car projected to the Y axis

\item {} 
\sphinxstyleliteralstrong{TAR\_velX} (\sphinxstyleliteralemphasis{float}) \textendash{} Target vehicle absolute velocity along the X axis

\item {} 
\sphinxstyleliteralstrong{TAR\_velY} (\sphinxstyleliteralemphasis{float}) \textendash{} Target vehicle absolute velocity along the Y axis

\item {} 
\sphinxstyleliteralstrong{TAR\_hdg} (\sphinxstyleliteralemphasis{float}) \textendash{} Target vehicle’s heading

\item {} 
\sphinxstyleliteralstrong{EGO\_velX} (\sphinxstyleliteralemphasis{float}) \textendash{} EGO car’s absolute velocity along the X axis

\item {} 
\sphinxstyleliteralstrong{EGO\_velY} (\sphinxstyleliteralemphasis{float}) \textendash{} EGO car’s absolute velocity along the Y axis

\item {} 
\sphinxstyleliteralstrong{EGO\_accX} (\sphinxstyleliteralemphasis{float}) \textendash{} EGO car’s absolute acceleration along the X axis

\item {} 
\sphinxstyleliteralstrong{EGO\_accY} (\sphinxstyleliteralemphasis{float}) \textendash{} EGO car’s absolute acceleration along the Y axis

\item {} 
\sphinxstyleliteralstrong{EGO\_hdg} (\sphinxstyleliteralemphasis{float}) \textendash{} EGO car’s heading

\end{itemize}

\end{description}\end{quote}

\end{fulllineitems}



\section{TrackPoint}
\label{\detokenize{datacontainers:trackpoint}}\index{TrackPoint (class in data\_containers)}

\begin{fulllineitems}
\phantomsection\label{\detokenize{datacontainers:data_containers.TrackPoint}}\pysiglinewithargsret{\sphinxstrong{class }\sphinxcode{data\_containers.}\sphinxbfcode{TrackPoint}}{\emph{mcc=0}, \emph{beam={[}{]}}, \emph{x=0}, \emph{y=0}, \emph{dx=0}, \emph{dy=0}, \emph{rvelocity=0}, \emph{razimuth=0}, \emph{rrange=0}}{}
Creates a point which is a part of the track.
\begin{quote}\begin{description}
\item[{Return type}] \leavevmode
object

\item[{Parameters}] \leavevmode\begin{itemize}
\item {} 
\sphinxstyleliteralstrong{mcc} (\sphinxstyleliteralemphasis{int}) \textendash{} The state of a Master-Clock-Cycle counter of the RADAR.

\item {} 
\sphinxstyleliteralstrong{beam} (\sphinxstyleliteralemphasis{int}) \textendash{} Identifies to which beam the detection has been assigned by the RADAR.

\item {} 
\sphinxstyleliteralstrong{x} (\sphinxstyleliteralemphasis{float}) \textendash{} X coordinate

\item {} 
\sphinxstyleliteralstrong{y} (\sphinxstyleliteralemphasis{float}) \textendash{} Y coordinate

\item {} 
\sphinxstyleliteralstrong{dx} (\sphinxstyleliteralemphasis{float}) \textendash{} the time derivative of x

\item {} 
\sphinxstyleliteralstrong{dy} (\sphinxstyleliteralemphasis{float}) \textendash{} the time derivative of y

\item {} 
\sphinxstyleliteralstrong{rvelocity} (\sphinxstyleliteralemphasis{float}) \textendash{} an absolute velocity as measured by RADAR

\item {} 
\sphinxstyleliteralstrong{razimuth} (\sphinxstyleliteralemphasis{float}) \textendash{} an azimuth as measured by RADAR

\end{itemize}

\end{description}\end{quote}
\index{get\_array() (data\_containers.TrackPoint method)}

\begin{fulllineitems}
\phantomsection\label{\detokenize{datacontainers:data_containers.TrackPoint.get_array}}\pysiglinewithargsret{\sphinxbfcode{get\_array}}{}{}
Returns x and y coordinates and their time derivatives in a 4-element numpy array
\begin{quote}\begin{description}
\item[{Returns}] \leavevmode
a numpy of the shape (4,1); np.array({[}x, dx, y, dy{]})

\end{description}\end{quote}

\end{fulllineitems}


\end{fulllineitems}



\section{DetectionList}
\label{\detokenize{datacontainers:detectionlist}}\index{DetectionList (class in data\_containers)}

\begin{fulllineitems}
\phantomsection\label{\detokenize{datacontainers:data_containers.DetectionList}}\pysigline{\sphinxstrong{class }\sphinxcode{data\_containers.}\sphinxbfcode{DetectionList}}
\end{fulllineitems}



\section{UnAssignedDetectionList}
\label{\detokenize{datacontainers:unassigneddetectionlist}}\index{UnAssignedDetectionList (class in data\_containers)}

\begin{fulllineitems}
\phantomsection\label{\detokenize{datacontainers:data_containers.UnAssignedDetectionList}}\pysiglinewithargsret{\sphinxstrong{class }\sphinxcode{data\_containers.}\sphinxbfcode{UnAssignedDetectionList}}{\emph{Tsampling}, \emph{gate}}{}~\begin{quote}\begin{description}
\item[{Parameters}] \leavevmode\begin{itemize}
\item {} 
\sphinxstyleliteralstrong{Tsampling} \textendash{} 

\item {} 
\sphinxstyleliteralstrong{gate} \textendash{} 

\end{itemize}

\end{description}\end{quote}
\index{new\_detection() (data\_containers.UnAssignedDetectionList method)}

\begin{fulllineitems}
\phantomsection\label{\detokenize{datacontainers:data_containers.UnAssignedDetectionList.new_detection}}\pysiglinewithargsret{\sphinxbfcode{new\_detection}}{\emph{detection}}{}
Tests whether or not the list of unassigned detections can form a new track.
\begin{quote}\begin{description}
\item[{Parameters}] \leavevmode
\sphinxstyleliteralstrong{detection} ({\hyperref[\detokenize{datacontainers:data_containers.DetectionPoint}]{\sphinxcrossref{\sphinxstyleliteralemphasis{DetectionPoint}}}}) \textendash{} The detection which is going to be tested.

\end{description}\end{quote}

\end{fulllineitems}

\index{two\_point\_projection() (data\_containers.UnAssignedDetectionList method)}

\begin{fulllineitems}
\phantomsection\label{\detokenize{datacontainers:data_containers.UnAssignedDetectionList.two_point_projection}}\pysiglinewithargsret{\sphinxbfcode{two\_point\_projection}}{\emph{det1}, \emph{det2}}{}
Extrapolates two detections in terms of the first order polynomial.
The extrapolation is computed from x,y coordinates.
\begin{quote}\begin{description}
\item[{Parameters}] \leavevmode\begin{itemize}
\item {} 
\sphinxstyleliteralstrong{start\_detection} ({\hyperref[\detokenize{datacontainers:data_containers.DetectionPoint}]{\sphinxcrossref{\sphinxstyleliteralemphasis{DetectionPoint}}}}) \textendash{} The first point of the extrapolation. X and Y coordinates are being used only.

\item {} 
\sphinxstyleliteralstrong{end\_detection} ({\hyperref[\detokenize{datacontainers:data_containers.DetectionPoint}]{\sphinxcrossref{\sphinxstyleliteralemphasis{DetectionPoint}}}}) \textendash{} The second point of the extrapolation. X and Y coordinates are being used only.

\end{itemize}

\item[{Returns}] \leavevmode
Projected point, X and Y coordinates are set only.

\item[{Return type}] \leavevmode
{\hyperref[\detokenize{datacontainers:data_containers.DetectionPoint}]{\sphinxcrossref{DetectionPoint}}}

\end{description}\end{quote}

\end{fulllineitems}


\end{fulllineitems}



\section{Track}
\label{\detokenize{datacontainers:track}}\index{Track (class in data\_containers)}

\begin{fulllineitems}
\phantomsection\label{\detokenize{datacontainers:data_containers.Track}}\pysiglinewithargsret{\sphinxstrong{class }\sphinxcode{data\_containers.}\sphinxbfcode{Track}}{\emph{trackID}}{}
\end{fulllineitems}



\chapter{Track Management Module}
\label{\detokenize{trackmanagement:track-management-module}}\label{\detokenize{trackmanagement::doc}}
The module contains one single class which manages all the necessary processes to organize all tracks, starts them from
the scratch and stops them accordingly. Here is being kept a list of current tracks, they are updated when a new detection is
assigned to them, incoming detections are sorted out and either assigned to already existing track or they are stored into
a list of unassigned detections.
\phantomsection\label{\detokenize{trackmanagement:module-track_management}}\index{track\_management (module)}

\section{TrackManager}
\label{\detokenize{trackmanagement:trackmanager}}
TrackManager is a class of which primary task is to organize life of all tracks starting by their initiation to
the moment when they are terminated. Advanced management as a parallel tracks merging is not implemented yet.

The main part of the TrackManager is a list of all active tracks where every track is a list itself, see
{\hyperref[\detokenize{datacontainers:data_containers.Track}]{\sphinxcrossref{\sphinxcode{data\_containers.Track()}}}} in a file {\hyperref[\detokenize{datacontainers::doc}]{\sphinxcrossref{\DUrole{doc}{data\_containers.py}}}} file.

Another list is a list of unassigned detections, {\hyperref[\detokenize{datacontainers:data_containers.UnAssignedDetectionList}]{\sphinxcrossref{\sphinxcode{data\_containers.UnAssignedDetectionList()}}}} defined in a
{\hyperref[\detokenize{datacontainers::doc}]{\sphinxcrossref{\DUrole{doc}{data\_containers.py}}}} file. A newly incoming detection if not assigned to any existing track is
stored here. Every newcomer is then tested if it can form a new track in combination with other detections already
in a list. If not they are removed from a list and ‘trashed’ after couple of cycles in order to free the memory and avoid
any possible false track initiation.
\index{TrackManager (class in track\_management)}

\begin{fulllineitems}
\phantomsection\label{\detokenize{trackmanagement:track_management.TrackManager}}\pysiglinewithargsret{\sphinxstrong{class }\sphinxcode{track\_management.}\sphinxbfcode{TrackManager}}{\emph{gate=None}, \emph{tracker\_type=\{‘filter\_type’: ‘kalman\_filter’}, \emph{’dim\_x’: 4}, \emph{’dim\_z’: 2\}}, \emph{Tsampling=0.05}}{}~\index{append\_track() (track\_management.TrackManager method)}

\begin{fulllineitems}
\phantomsection\label{\detokenize{trackmanagement:track_management.TrackManager.append_track}}\pysiglinewithargsret{\sphinxbfcode{append\_track}}{\emph{track}}{}
A new track is being appended to the list of tracks, a new tracking filter is also created
alongside the track and is assigned to it.
\begin{quote}\begin{description}
\item[{Parameters}] \leavevmode
\sphinxstyleliteralstrong{track} ({\hyperref[\detokenize{datacontainers:data_containers.Track}]{\sphinxcrossref{\sphinxstyleliteralemphasis{Track}}}}) \textendash{} 

\end{description}\end{quote}

\end{fulllineitems}


\end{fulllineitems}



\chapter{Tracking Filters Module}
\label{\detokenize{trackingfilters::doc}}\label{\detokenize{trackingfilters:tracking-filters-module}}
The module contains two classes.
\begin{enumerate}
\item {} 
\sphinxstylestrong{EKF} - to produce an extended Kalman filter,

\item {} 
\sphinxstylestrong{Fading Filter} - to produce a fading memory filter.

\end{enumerate}

These filters were developed based on the tutorial in
.. \_a link: \sphinxurl{https://pykalman.github.io/}
\phantomsection\label{\detokenize{trackingfilters:module-tracking_filters}}\index{tracking\_filters (module)}

\section{Tracking Filter}
\label{\detokenize{trackingfilters:tracking-filter}}\index{TrackingFilter (class in tracking\_filters)}

\begin{fulllineitems}
\phantomsection\label{\detokenize{trackingfilters:tracking_filters.TrackingFilter}}\pysiglinewithargsret{\sphinxstrong{class }\sphinxcode{tracking\_filters.}\sphinxbfcode{TrackingFilter}}{\emph{dim\_x}, \emph{dim\_z}, \emph{dim\_u=0}}{}~\begin{quote}\begin{description}
\item[{Parameters}] \leavevmode\begin{itemize}
\item {} 
\sphinxstyleliteralstrong{dim\_x} \textendash{} 

\item {} 
\sphinxstyleliteralstrong{dim\_z} \textendash{} 

\item {} 
\sphinxstyleliteralstrong{dim\_u} \textendash{} 

\end{itemize}

\end{description}\end{quote}

\end{fulllineitems}



\section{Kalman Filter}
\label{\detokenize{trackingfilters:kalman-filter}}\index{KalmanFilter (class in tracking\_filters)}

\begin{fulllineitems}
\phantomsection\label{\detokenize{trackingfilters:tracking_filters.KalmanFilter}}\pysiglinewithargsret{\sphinxstrong{class }\sphinxcode{tracking\_filters.}\sphinxbfcode{KalmanFilter}}{\emph{dim\_x}, \emph{dim\_z}, \emph{dim\_u=0}}{}
\end{fulllineitems}



\chapter{Radar Plots Module}
\label{\detokenize{radarplots:radar-plots-module}}\label{\detokenize{radarplots::doc}}
The module contains set of functions which plot results from radar data pre-processing and tracking.
\phantomsection\label{\detokenize{radarplots:module-radar_plots}}\index{radar\_plots (module)}

\section{static\_plot\_grid\_hist\_selections}
\label{\detokenize{radarplots:static-plot-grid-hist-selections}}\index{static\_plot\_grid\_hist\_selections() (in module radar\_plots)}

\begin{fulllineitems}
\phantomsection\label{\detokenize{radarplots:radar_plots.static_plot_grid_hist_selections}}\pysiglinewithargsret{\sphinxcode{radar\_plots.}\sphinxbfcode{static\_plot\_grid\_hist\_selections}}{}{}
\end{fulllineitems}



\section{static\_plot\_selections}
\label{\detokenize{radarplots:static-plot-selections}}\index{static\_plot\_selections() (in module radar\_plots)}

\begin{fulllineitems}
\phantomsection\label{\detokenize{radarplots:radar_plots.static_plot_selections}}\pysiglinewithargsret{\sphinxcode{radar\_plots.}\sphinxbfcode{static\_plot\_selections}}{}{}
\end{fulllineitems}



\section{static\_plotREF\_selections module}
\label{\detokenize{radarplots:static-plotref-selections-module}}\index{static\_plotREF\_selections() (in module radar\_plots)}

\begin{fulllineitems}
\phantomsection\label{\detokenize{radarplots:radar_plots.static_plotREF_selections}}\pysiglinewithargsret{\sphinxcode{radar\_plots.}\sphinxbfcode{static\_plotREF\_selections}}{}{}
\end{fulllineitems}



\chapter{Utilities Package}
\label{\detokenize{utils::doc}}\label{\detokenize{utils:utilities-package}}
The package contains helpful functions which solve different linear-algebra problems.
\phantomsection\label{\detokenize{utils:module-utils}}\index{utils (module)}

\chapter{Indices and tables}
\label{\detokenize{index:indices-and-tables}}\begin{itemize}
\item {} 
\DUrole{xref,std,std-ref}{genindex}

\item {} 
\DUrole{xref,std,std-ref}{modindex}

\item {} 
\DUrole{xref,std,std-ref}{search}

\end{itemize}


\renewcommand{\indexname}{Python Module Index}
\begin{sphinxtheindex}
\def\bigletter#1{{\Large\sffamily#1}\nopagebreak\vspace{1mm}}
\bigletter{d}
\item {\sphinxstyleindexentry{data\_containers}}\sphinxstyleindexpageref{datacontainers:\detokenize{module-data_containers}}
\indexspace
\bigletter{r}
\item {\sphinxstyleindexentry{radar\_plots}}\sphinxstyleindexpageref{radarplots:\detokenize{module-radar_plots}}
\indexspace
\bigletter{t}
\item {\sphinxstyleindexentry{track\_management}}\sphinxstyleindexpageref{trackmanagement:\detokenize{module-track_management}}
\item {\sphinxstyleindexentry{tracking\_filters}}\sphinxstyleindexpageref{trackingfilters:\detokenize{module-tracking_filters}}
\indexspace
\bigletter{u}
\item {\sphinxstyleindexentry{utils}}\sphinxstyleindexpageref{utils:\detokenize{module-utils}}
\end{sphinxtheindex}

\renewcommand{\indexname}{Index}
\printindex
\end{document}